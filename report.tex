\documentclass[a4paper,12pt]{article}
\usepackage[T2A]{fontenc}
\usepackage[utf8]{inputenc}
\usepackage[russian]{babel}
\usepackage[style=russian]{csquotes}
\usepackage{xcolor}
\usepackage{amsmath}
\usepackage{amsfonts}
\usepackage{tikz}
\usepackage{pgfplots}
\usepackage[l3]{csvsimple}

\usepackage{cleveref}
\crefformat{equation}{(#2#1#3)}
\crefrangeformat{equation}{(#3#1#4)~-~(#5#2#6)}
\crefmultiformat{equation}{(#2#1#3)}{ и~(#2#1#3)}{, (#2#1#3)}{ и~(#2#1#3)}

\allowdisplaybreaks

\patchcmd{\thebibliography}{\section*{Список литературы}}{}{}{}

\newcommand{\UpdateMe}[1]{\textcolor{red}{#1}}
\DeclareMathOperator*{\argmax}{argmax}

\newcommand{\University}{Московский государственный университет имени М.~В.~Ломоносова}
\newcommand{\Department}{Кафедра \UpdateMe{НАЗВАНИЕ-КАФЕДРЫ}}
\newcommand{\Student}{\UpdateMe{ИМЯ-СТУДЕНТА}}
\newcommand{\GroupNum}{\UpdateMe{НОМЕР}}
\newcommand{\Seminar}{\UpdateMe{НАЗВАНИЕ-ПРАКТИКУМА}}

\begin{document}
% \begin{titlepage}
%   \centering
%   {\scshape\Large \University\par}\vspace{1cm}{\scshape\large \Department\par}
%   \vfill
%   {\huge\bfseries ОТЧЕТ\par}{\Largeпо задаче практикума \enquote{\Seminar}\par}
%   \vfill
%   \hfill\begin{minipage}{0.45\linewidth}Выполнил студент гр. \GroupNum:\\\Student\end{minipage}
%   \vfill
%   {\large Москва, \the\year{}\par}
% \end{titlepage}

\section*{Задание}
Найти решение задачи, удовлетворяющее указанным условиям ({\bfseries Задача~3.1}):
\begin{gather*}
  \frac{\partial u}{\partial t}=\alpha \frac{\partial^2 u}{\partial x^2} + u^4,\\
  \left.\frac{\partial u}{\partial x}\right\vert_{x=0}=0,\\
  \left.\frac{\partial u}{\partial x}\right\vert_{x=1}=-50u^4,\\
  \left.u\right\vert_{t=0}=\beta(1-x^2)^2,\\
  x \in [0, 1], \quad t \in [0, 1], \quad \alpha \in \{0.1, 1.0\}, \quad \beta \in \{1.0, 0.1\}
\end{gather*}
и вычислить значения следующих функционалов:
\begin{gather*}
  f_1(t) = \int_{0}^{1}{u(x, t) dx},\\
  f_2(t) = \int_{0}^{1}{u^4(x, t) dx} - 50\alpha\left.u^4\right\vert_{x=1}.
\end{gather*}

\section*{Методика решения}

Введем равномерную сетку \(\Omega=\{(x_{i}, t_{j})=(ih, j\tau): i \in 0..N-1 \quad j\in 0..M-1\}\), где \(h=1/(N-1)\) и \(\tau=1/(M-1)\).\\
Будем искать решение в виде сеточной функции, определенной в каждом узле сетки \(u(x_{i},t_{j})=u_{i}^{j}\).

\subsection*{Разностная схема}
Будем использовать модицифированную схему Кранка-Николсона, которая имеет следующий вид:
\begin{multline}\label{scheme}
  \frac{u_{i}^{j+1}-u_{i}^{j}}{\tau}=\alpha\left(\frac{u_{i+1}^{j+1}-2u_{i}^{j+1}+u_{i-1}^{j+1}}{2h^2}+\frac{u_{i+1}^{j}-2u_{i}^{j}+u_{i-1}^{j}}{2h^2}\right) + \\
  + \left(2u_{i}^{j+1}-u_{i}^{j}\right)\left(u_{i}^{j}\right)^3.
\end{multline}

\subsection*{Порядок аппроксимации}

Разложим входящие в уравнение \cref{scheme} значения сеточной функции в ряд Тейлора в точке \((x_i, t_j)\):
\begin{gather*}
  u_{i\pm1}^{j}=u(x_{i}\pm h, t_{j})=u \pm h u_x +\frac{1}{2} h^2 u_{xx} \pm \frac{1}{6} h^3 u_{xxx} + O(h^{4}),\\
  u_{i}^{j+1}=u(x_{i}, t_{j}+\tau)=u + \tau u_t +\frac{1}{2} \tau^2 u_{tt} \pm \frac{1}{6} \tau^3 u_{ttt} + O(\tau^{4}),\\
  \begin{split}
    u_{i\pm1}^{j+1}=u(x_{i}\pm h,t_{j}+\tau)=u+\tau  u_t+\frac{1}{2} \tau ^2 u_{tt}\pm h u_x+\frac{1}{2} h^2 u_{xx}\pm h \tau  u_{xt}\pm \frac{1}{2} h \tau ^2 u_{xtt}+\\
    +\frac{1}{2} h^2 \tau  u_{xxt}\pm \frac{1}{6} h^3 u_{xxx}+\frac{1}{4} h^2 \tau ^2 u_{xxtt}\pm \frac{1}{6} h^3 \tau  u_{xxxt}+\frac{1}{24} h^4 u_{xxxx}\pm \frac{1}{12} h^3 \tau ^2 u_{xxxtt}+\\
    +\frac{1}{24} h^4 \tau  u_{xxxxt}+O(h^4, \tau^2)
  \end{split}
\end{gather*}

Подставив данные выражения в \cref{scheme} получим
\begin{equation*}
  u_{t}+\frac{\tau}{2} u_{tt} = \alpha u_{xx} + \frac{\tau}{2} \alpha u_{xxt} + u^4 + 2 \tau u^3 u_{t} + O(h^2, \tau^2).
\end{equation*}
Выразив \(u_{t}\) и \(u_{tt}\) из исходного уравнения, с учетом \(u^3 u_{t}=\frac{1}{4}\left(u^4\right)_t\) получим, что порядок аппроксимации равен \(O(h^2, \tau^2)\).

\subsection*{Устойчивость}

Рассмотрим значения в узлах сетки как \(u_{i}^{j} = \bar{u}_{i}^{j} + \delta_{i}^{j}\), где \(\bar{u}\) - истинное решение, а \(\delta\) - ошибка. Подставим данные выражения в уравнение \cref{scheme} и рассмотрим уравнение относительно ошибок. Будучи линеаризованным оно примет вид:
\begin{multline*}
  \frac{\delta _{i}^{j+1}-\delta _{i}^{j}}{\tau }=\alpha  \left(\frac{\delta _{i-1}^{j}-2 \delta _{i}^{j}+\delta _{i+1}^{j}}{2 h^2}+\frac{\delta _{i-1}^{j+1}-2 \delta _{i}^{j+1}+\delta
    _{i+1}^{j+1}}{2 h^2}\right)+ \\
  + \left(2 \delta _{i}^{j+1}-\delta _{i}^{j}\right) \left(u_{i}^{j}\right)^3+3 \delta _{i}^{j} \left(2 u_{i}^{j+1}-u_{i}^{j}\right) \left(u_{i}^{j}\right)^2
\end{multline*}

Выполним замену \(\delta_{m}^{n}=\lambda^{n} e^{i m \phi}\), после чего поделим обе части на \(\lambda^{n} e^{i m \phi}\):
\begin{equation*}
  \frac{\lambda -1}{\tau }=\frac{\alpha  (\lambda +1) e^{-i \phi } \left(-1+e^{i \phi }\right)^2}{2 h^2}+(2 \lambda -1) \left(u_{m}^{n}\right){}^3+3 \left(2
  u_{m}^{n+1}-u_{m}^{n}\right) \left(u_{m}^{n}\right){}^2.
\end{equation*}
Выразив \(\lambda\) получим
\begin{equation*}
  \lambda = 1 + 2 \tau \frac{-\alpha  \cos (\phi )+\alpha +h^2 \left(u_{m}^{n}-3 u_{m}^{n+1}\right) \left(u_{m}^{n}\right){}^2}{\alpha  \tau  \cos (\phi )-\alpha  \tau +2 h^2 \tau
    \left(u_{m}^{n}\right)^3-h^2},
\end{equation*}
то есть \(\left|\lambda\right| < 1 + O(\tau)\), и следовательно схема устойчива.

\subsection*{Система уравнений}

Выразим из уравнения \cref{scheme} неизвестные \(u_{*}^{j+1}\), а так же обозначим
\begin{gather*}
  A = C = -\frac{\alpha}{2h^2},\\
  B_{i}^{j} = \frac{1}{\tau}-\frac{\alpha}{h^2} - 2 \left(u_{i}^{j}\right)^3,\\
  \hat{B}_{i}^{j} = \frac{1}{\tau}+\frac{\alpha}{h^2} - \left(u_{i}^{j}\right)^3,
\end{gather*}
что даст нам короткую запись
\begin{equation}\label{eq}
  A u_{i-1}^{j+1} + B_{i}^{j} u_{i}^{j+1} + C u_{i+1}^{j+1} = -A u_{i-1}^{j} + \hat{B}_{i}^{j} u_{i}^{j} - C u_{i+1}^{j}=f_{i}^{j}.
\end{equation}

Введем допольнительно фиктивные узлы \(u_{-1}^{*}\) и \(u_{N}^{*}\), которые будут входить в определение граничных условий. Для симметричной записи обозначим \(n=N-1\) и рассмотрим подробнее второе граничное условие: \(\frac{\partial u}{\partial x}+50u^4=0\) при \(x=1\). Для его аппроксимации будем использовать сразу две разностные схемы, одна из которых свяжет два временных слоя:
\begin{gather}
  \label{rcond_linear}\frac{u_{n+1}^{j+1}-u_{n-1}^{j+1}}{2h}+50\left(4u_{n}^{j+1}-3u_{n}^{j}\right)\left(u_{n}^{j}\right)^3=0,\\
  \label{rcond}\frac{u_{n+1}^{j}-u_{n-1}^{j}}{2h}+50 \left(u_{n}^{j}\right)^4 =0.
\end{gather}
Данные разностные схемы сохраняют порядок аппроксимации \(O(h^2,\
tau^2)\), причём уравнение \cref{rcond_linear} линйено относительно \(u_{*}^{j+1}\).

Для первого граничного условия имеем
\begin{equation}\label{lcond}
  \frac{u_{1}^{j+1}-u_{-1}^{j+1}}{2h}=0.
\end{equation}


Окончательно, подставив \(u_{-1}^{j+1}\) из \cref{lcond} в \cref{eq} при \(i=0\) и \(u_{n+1}^{j+1}\) из \cref{rcond_linear} и \(u_{n+1}^{j}\) из \label{rcond} в \cref{eq} при \(i=n=N-1\), получаем
\begin{equation*}
  \begin{cases}
    B_{0}^{j} u_{0}^{j+1} + \left(A + C\right) u_{1}^{j+1} = f_{0}^{j},                     \\
    A u_{i-1}^{j+1} + B_{i}^{j} u_{i}^{j+1} + C u_{i+1}^{j+1} =  f_{i}^{j}, \quad i=1..N-2, \\
    \left(A + C\right) u_{N-2}^{j+1} + \left(B_{N-1}^{j} - 400 h C \left(u_{N-1}^{j}\right)^3 \right) u_{N-1}^{j+1} = \hat{f}_{N-1}^{j},
  \end{cases}
\end{equation*}
где \(\hat{f}_{N-1}^{j}=f_{N-1}^{j}-200 h C \left(u_{N-1}^{j}\right)^4\).

Матрица данной системы является трехдиагональной и допускает решение методом прогонки.

\section*{Результаты}
Были рассмотрены следующие варианты сеток: \(M=N=51\), \(M=N=101\) и \(M=N=201\).

\subsection*{Случай \(\alpha=0.1\) и \(\beta=0.1\)}

\subsubsection*{Размеры \(M=N=51\)}

\begin{tikzpicture}
  \begin{axis}[width=8cm, height=6cm, legend pos=north west, grid = major, grid style={dashed, gray!30},
      yticklabel style={/pgf/number format/fixed, /pgf/number format/precision=4}]
    \addplot [thick, mark = none, red]table [x=t, y=f1, col sep=comma] {fs_a0.1_b0.1_N51_M51.csv};
    \addlegendentry{$f_1$}
  \end{axis}
\end{tikzpicture}
\begin{tikzpicture}
  \begin{axis}[width=8cm, height=6cm, legend pos=north east, grid = major, grid style={dashed, gray!30}]
    \addplot [thick, mark = none, blue]table [x=t, y=f2, col sep=comma] {fs_a0.1_b0.1_N51_M51.csv};
    \addlegendentry{$f_2$}
  \end{axis}
\end{tikzpicture}

Ниже приведены некоторые значения функции \(u\):

\csvautotabular{u_a0.1_b0.1_N51_M51.csv}

\subsubsection*{Размеры \(M=N=101\)}

\begin{tikzpicture}
  \begin{axis}[width=8cm, height=6cm, legend pos=north west, grid = major, grid style={dashed, gray!30},
      yticklabel style={/pgf/number format/fixed, /pgf/number format/precision=4}]
    \addplot [thick, mark = none, red]table [x=t, y=f1, col sep=comma] {fs_a0.1_b0.1_N101_M101.csv};
    \addlegendentry{$f_1$}
  \end{axis}
\end{tikzpicture}
\begin{tikzpicture}
  \begin{axis}[width=8cm, height=6cm, legend pos=north east, grid = major, grid style={dashed, gray!30}]
    \addplot [thick, mark = none, blue]table [x=t, y=f2, col sep=comma] {fs_a0.1_b0.1_N101_M101.csv};
    \addlegendentry{$f_2$}
  \end{axis}
\end{tikzpicture}

Ниже приведены некоторые значения функции \(u\):

\csvautotabular{u_a0.1_b0.1_N101_M101.csv}

\subsubsection*{Размеры \(M=N=201\)}

\begin{tikzpicture}
  \begin{axis}[width=8cm, height=6cm, legend pos=north west, grid = major, grid style={dashed, gray!30},
      yticklabel style={/pgf/number format/fixed, /pgf/number format/precision=4}]
    \addplot [thick, mark = none, red]table [x=t, y=f1, col sep=comma] {fs_a0.1_b0.1_N201_M201.csv};
    \addlegendentry{$f_1$}
  \end{axis}
\end{tikzpicture}
\begin{tikzpicture}
  \begin{axis}[width=8cm, height=6cm, legend pos=north east, grid = major, grid style={dashed, gray!30}]
    \addplot [thick, mark = none, blue]table [x=t, y=f2, col sep=comma] {fs_a0.1_b0.1_N201_M201.csv};
    \addlegendentry{$f_2$}
  \end{axis}
\end{tikzpicture}

Ниже приведены некоторые значения функции \(u\):

\csvautotabular{u_a0.1_b0.1_N201_M201.csv}

\subsubsection*{Таблица отношений}
Ниже представлена таблица, с отношением значений сеточной функции при различных параметрах сеток \(\left(\frac{u_{M=101,N=101}-u_{M=51,N=51}}{u_{M=201,N=201}-u_{M=101,N=101}}\right)\):

\csvautotabular{rel_a0.1_b0.1.csv}

\subsection*{Случай \(\alpha=1.0\) и \(\beta=0.1\)}

\subsubsection*{Размеры \(M=N=51\)}

\begin{tikzpicture}
  \begin{axis}[width=8cm, height=6cm, legend pos=south west, grid = major, grid style={dashed, gray!30}]
    \addplot [thick, mark = none, red]table [x=t, y=f1, col sep=comma] {fs_a1.0_b0.1_N51_M51.csv};
    \addlegendentry{$f_1$}
  \end{axis}
\end{tikzpicture}
\begin{tikzpicture}
  \begin{axis}[width=8cm, height=6cm, legend pos=north east, grid = major, grid style={dashed, gray!30}]
    \addplot [thick, mark = none, blue]table [x=t, y=f2, col sep=comma] {fs_a1.0_b0.1_N51_M51.csv};
    \addlegendentry{$f_2$}
  \end{axis}
\end{tikzpicture}

Ниже приведены некоторые значения функции \(u\):

\csvautotabular{u_a1.0_b0.1_N51_M51.csv}

\subsubsection*{Размеры \(M=N=101\)}

\begin{tikzpicture}
  \begin{axis}[width=8cm, height=6cm, legend pos=south west, grid = major, grid style={dashed, gray!30}]
    \addplot [thick, mark = none, red]table [x=t, y=f1, col sep=comma] {fs_a1.0_b0.1_N101_M101.csv};
    \addlegendentry{$f_1$}
  \end{axis}
\end{tikzpicture}
\begin{tikzpicture}
  \begin{axis}[width=8cm, height=6cm, legend pos=north east, grid = major, grid style={dashed, gray!30}]
    \addplot [thick, mark = none, blue]table [x=t, y=f2, col sep=comma] {fs_a1.0_b0.1_N101_M101.csv};
    \addlegendentry{$f_2$}
  \end{axis}
\end{tikzpicture}

Ниже приведены некоторые значения функции \(u\):

\csvautotabular{u_a1.0_b0.1_N101_M101.csv}

\subsubsection*{Размеры \(M=N=201\)}

\begin{tikzpicture}
  \begin{axis}[width=8cm, height=6cm, legend pos=south west, grid = major, grid style={dashed, gray!30}]
    \addplot [thick, mark = none, red]table [x=t, y=f1, col sep=comma] {fs_a1.0_b0.1_N201_M201.csv};
    \addlegendentry{$f_1$}
  \end{axis}
\end{tikzpicture}
\begin{tikzpicture}
  \begin{axis}[width=8cm, height=6cm, legend pos=north east, grid = major, grid style={dashed, gray!30}]
    \addplot [thick, mark = none, blue]table [x=t, y=f2, col sep=comma] {fs_a1.0_b0.1_N201_M201.csv};
    \addlegendentry{$f_2$}
  \end{axis}
\end{tikzpicture}

Ниже приведены некоторые значения функции \(u\):

\csvautotabular{u_a1.0_b0.1_N201_M201.csv}

\subsubsection*{Таблица отношений}
Ниже представлена таблица, с отношением значений сеточной функции при различных параметрах сеток \(\left(\frac{u_{M=101,N=101}-u_{M=51,N=51}}{u_{M=201,N=201}-u_{M=101,N=101}}\right)\):

\csvautotabular{rel_a1.0_b0.1.csv}

\subsection*{Случай \(\alpha=1.0\) и \(\beta=1.0\)}

\subsubsection*{Размеры \(M=N=51\)}

\begin{tikzpicture}
  \begin{axis}[width=8cm, height=6cm, legend pos=south west, grid = major, grid style={dashed, gray!30}]
    \addplot [thick, mark = none, red]table [x=t, y=f1, col sep=comma] {fs_a1.0_b1.0_N51_M51.csv};
    \addlegendentry{$f_1$}
  \end{axis}
\end{tikzpicture}
\begin{tikzpicture}
  \begin{axis}[width=8cm, height=6cm, legend pos=north east, grid = major, grid style={dashed, gray!30}]
    \addplot [thick, mark = none, blue]table [x=t, y=f2, col sep=comma] {fs_a1.0_b1.0_N51_M51.csv};
    \addlegendentry{$f_2$}
  \end{axis}
\end{tikzpicture}

Ниже приведены некоторые значения функции \(u\):

\csvautotabular{u_a1.0_b1.0_N51_M51.csv}

\subsubsection*{Размеры \(M=N=101\)}

\begin{tikzpicture}
  \begin{axis}[width=8cm, height=6cm, legend pos=south west, grid = major, grid style={dashed, gray!30}]
    \addplot [thick, mark = none, red]table [x=t, y=f1, col sep=comma] {fs_a1.0_b1.0_N101_M101.csv};
    \addlegendentry{$f_1$}
  \end{axis}
\end{tikzpicture}
\begin{tikzpicture}
  \begin{axis}[width=8cm, height=6cm, legend pos=north east, grid = major, grid style={dashed, gray!30}]
    \addplot [thick, mark = none, blue]table [x=t, y=f2, col sep=comma] {fs_a1.0_b1.0_N101_M101.csv};
    \addlegendentry{$f_2$}
  \end{axis}
\end{tikzpicture}

Ниже приведены некоторые значения функции \(u\):

\csvautotabular{u_a1.0_b1.0_N101_M101.csv}

\subsubsection*{Размеры \(M=N=201\)}

\begin{tikzpicture}
  \begin{axis}[width=8cm, height=6cm, legend pos=south west, grid = major, grid style={dashed, gray!30}]
    \addplot [thick, mark = none, red]table [x=t, y=f1, col sep=comma] {fs_a1.0_b1.0_N201_M201.csv};
    \addlegendentry{$f_1$}
  \end{axis}
\end{tikzpicture}
\begin{tikzpicture}
  \begin{axis}[width=8cm, height=6cm, legend pos=north east, grid = major, grid style={dashed, gray!30}]
    \addplot [thick, mark = none, blue]table [x=t, y=f2, col sep=comma] {fs_a1.0_b1.0_N201_M201.csv};
    \addlegendentry{$f_2$}
  \end{axis}
\end{tikzpicture}

Ниже приведены некоторые значения функции \(u\):

\csvautotabular{u_a1.0_b1.0_N201_M201.csv}

\subsubsection*{Таблица отношений}
Ниже представлена таблица, с отношением значений сеточной функции при различных параметрах сеток \(\left(\frac{u_{M=101,N=101}-u_{M=51,N=51}}{u_{M=201,N=201}-u_{M=101,N=101}}\right)\):

\csvautotabular{rel_a1.0_b1.0.csv}

% \addcontentsline{toc}{section}{Список литературы}
% \begin{thebibliography}{99}
%   \bibitem{RKM}\textbf{О.Б. Арушанян, С.Ф. Залеткин} Решение систем обыкновенных дифференциальных уравнений методами Рунге--Кутта.
%   \bibitem{TAC}\textbf{Д.П. Ким} Теория автоматического управления. Том 2.
%   \bibitem{OM}\textbf{Н.Л. Майорова, Д.В. Глазков} Методы оптимизации.
%   \bibitem{DEP}\textbf{А.Ф. Филиппов} Сборник задач по дифференциальным уравнениям.
% \end{thebibliography}
\end{document}