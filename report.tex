\documentclass[a4paper,12pt]{article}
\usepackage[T2A]{fontenc}
\usepackage[utf8]{inputenc}
\usepackage[russian]{babel}
\usepackage[style=russian]{csquotes}
\usepackage{xcolor}
\usepackage{amsmath}
\usepackage{tikz}
\usepackage{pgfplots}
\usepackage[l3]{csvsimple}

\usepackage{cleveref}
\crefformat{equation}{(#2#1#3)}
\crefrangeformat{equation}{(#3#1#4)~-~(#5#2#6)}
\crefmultiformat{equation}{(#2#1#3)}{ и~(#2#1#3)}{, (#2#1#3)}{ и~(#2#1#3)}

\allowdisplaybreaks

\patchcmd{\thebibliography}{\section*{Список литературы}}{}{}{}

\newcommand{\UpdateMe}[1]{\textcolor{red}{#1}}
\DeclareMathOperator*{\argmax}{argmax}

\newcommand{\University}{Московский государственный университет имени М.~В.~Ломоносова}
\newcommand{\Department}{Кафедра \UpdateMe{НАЗВАНИЕ-КАФЕДРЫ}}
\newcommand{\Student}{\UpdateMe{ИМЯ-СТУДЕНТА}}
\newcommand{\GroupNum}{\UpdateMe{НОМЕР}}
\newcommand{\Seminar}{\UpdateMe{НАЗВАНИЕ-ПРАКТИКУМА}}

\begin{document}
% \begin{titlepage}
%   \centering
%   {\scshape\Large \University\par}\vspace{1cm}{\scshape\large \Department\par}
%   \vfill
%   {\huge\bfseries ОТЧЕТ\par}{\Largeпо задаче практикума \enquote{\Seminar}\par}
%   \vfill
%   \hfill\begin{minipage}{0.45\linewidth}Выполнил студент гр. \GroupNum:\\\Student\end{minipage}
%   \vfill
%   {\large Москва, \the\year{}\par}
% \end{titlepage}

\section*{Задание}
Найти решение задачи, удовлетворяющее указанным условиям ({\bfseries Задача~2.44}):
\begin{gather*}
  x_{tt}+x^2+\lambda t^2=1, \quad t\in(0, 1),\\
  \left.x\right\vert_{t=0}=0, \quad \left.x_{t}\right\vert_{t=0}=1,\\
  \left.x\right\vert_{t=1}=1.
\end{gather*}

\section*{Постановка задачи}

Введем обозначения \(x_1 = x, x_2=\dot{x}\). Тогда задача перепишется в виде:
\begin{gather}
  \label{common_system} \begin{cases}\dot{x}_2=1-x_1^2-\lambda t^2\\ \dot{x}_1=x_2 \end{cases},\\
  \label{left_conditions} x_1(0) = 0, \quad x_2(0)=1,\\
  \label{right_conditions} x_1(1) = 0.
\end{gather}

\section*{Методика решения}

\subsection*{Метод Рунге-Кутта}

Пусть дана задача Коши
\begin{equation*}
  \begin{cases} \dot{x}_1=f_1(x_1,x_2,t)\\ \dot{x}_2=f_2(x_1,x_2,t) \end{cases}
\end{equation*}
с известными в начальный момент времени значениями \(x_1^0\) и \(\ x_2^0\).

Допустим уже найдены значения в точке \(t_i\): \(x_1^{i}\) и \(\ x_2^{i}\). Тогда значения в точке \(t_{i+1}\) вычисляются по
схеме:

\begin{gather*}
  k_{m1}=h f_m(x_1^{i},x_2^{i}, t), \\ 
  k_{m2}=h f_m(x_1^{i} + 0.5 k_{11},x_2^{i} + 0.5 k_{21}, t + 0.5 h), \\ 
  k_{m3}=h f_m(x_1^{i} + 0.25 (k_{11}+k_{12}),x_2^{i} + 0.25 (k_{21}+k_{22}), t + 0.5 h),\\
  k_{m4}=f_m(x_1^{i} - k_{12} + 2 k_{13},x_2^{i} - k_{22} + 2 k_{23}, t + h),\\
  \begin{align*}
    k_{m5}=h f_m(x_1^{i} + \frac{1}{27}(7 k_{11}+10 k_{12}+k_{14}), x_2^{i} + \frac{1}{27}(7 k_{21}+10 k_{22}+k_{24}), t + \frac{2}{3} h),
  \end{align*} \\
  \begin{align*}
    k_{m6}=h f_m( & x_1^{i} + \frac{1}{625}(28 k_{11}-125 k_{12}+ 546 k_{13}+54 k_{14}-378 k_{15}),                      \\
                  & x_2^{i} + \frac{1}{625}(28 k_{21}-125 k_{22}+ 546 k_{23}+54 k_{24}-378 k_{25}), t + \frac{1}{5} h),
  \end{align*} \\
  x_1^{i+1}=\gamma_1^h(x_1^{i})=x_1^{i} + \frac{1}{24}k_{11} + \frac{5}{48} k_{14} + \frac{27}{56} k_{15} + \frac{125}{336} k_{16}, \\ 
  x_2^{i+1}=\gamma_2^h(x_2^{i})=x_2^{i} + \frac{1}{24}k_{21} + \frac{5}{48} k_{24} + \frac{27}{56} k_{25} + \frac{125}{336} k_{26},
\end{gather*}

где \(h=h(t_i)\) - адаптивный шаг по времени.

\subsection*{Адаптивный шаг по времени}

Пусть задан уровень погрешности \(\varepsilon\). Обозначим за \(y^{i}\)
значение искомой функции в текущей точке, а за \(h\) - текущий шаг по
времени.

Погрешность \(m\)-ой функции \(\gamma_{m}\) вычисляется с помощью контрольного члена:
\begin{equation*}
  \delta\gamma_{m} = \frac{-1}{336}\left(42 k_{m1}+224k_{m3}+21k_{m4}-162 k_{m5}-125 k_{m6}\right)
\end{equation*}
Тогда новый шаг на каждой итерации будет вычисляться по формулам
\begin{equation*}
  err = \Vert (\delta\gamma_{1}, \delta\gamma_{2})\Vert,
\end{equation*}
\begin{equation*}
  h'=\begin{cases}
    0.95 h \sqrt[5]{\frac{\varepsilon}{x+err}}, \quad err \not\in [0.1 \varepsilon, \varepsilon] \\
    h, \quad  err \in [0.1 \varepsilon, \varepsilon]
  \end{cases},
\end{equation*}
где \(x\) - маленькая
положительная константа избавляющая нас от необходимости проверять, что
\(err \neq 0\). Например можно взять \(x=0.001 \varepsilon\).

\subsection*{Метод стрельбы}

Для краевой задачи \cref{common_system} из краевого условия \cref{left_conditions} имеем
\begin{equation*}
  x_1^0=0, \quad x_2^0=1.
\end{equation*}

Параметр \(\lambda\) выберем в качестве параметра пристрелки.

Тогда решая систему \cref{common_system} с такими начальными условиями до момента времени \(t=1\) получим некоторое решение \(\tilde{x}_1\) и \(\tilde{x}_2\).

Варьируя параметр \(\lambda\) необходимо добиться выполнения краевого условия \cref{right_conditions}:
\begin{equation*}
  \tilde{x}_1(1)=x_1^N=0.
\end{equation*}

Будем искать параметр \(\lambda\) методом простых итераций. Введём функцию \(\phi\) показывающую отклонение от заданного
граничного условия:
\begin{equation*}
  \phi(\lambda)=x_1^N-0.
\end{equation*}

Разложив эту функцию в окрестности точки \((a^{i})\) получим:

\begin{equation*}
  \phi(\lambda)\simeq\phi(\lambda^{i})+\delta \lambda\frac{\partial \phi}{\partial \lambda}(\lambda^{i})\simeq0,
\end{equation*}

где \(\delta \lambda=\lambda^{i+1}-\lambda^{i}\). 

Тогда итерационный процесс может быть выражен формулой:
\begin{equation*}
  \lambda^{i+1}=\lambda^{i}-\phi(\lambda^{i}) / \phi_\lambda(\lambda^{i}).
\end{equation*}

В случае \(\phi_\lambda(\lambda^{i})=0\) предполагается использовать иную стартовую точку.

\section*{Результаты}

\subsection*{Сводная таблица}
\csvautotabular{data/stats.csv}

Для таблицы выше должно быть выполнено:
\begin{equation*}
  \frac{n_{i}}{n_{j}}=\left(\frac{\varepsilon_{i}}{\varepsilon_{j}}\right)^{1/s}, \quad \frac{error_{i}}{error_{j}}=\left(\frac{\varepsilon_{j}}{\varepsilon_{i}}\right)^{(1-1/s)},
\end{equation*}
где \(s\) - порядок метода Рунге--Кутта.

Для метода Рунге--Кутта 5-го порядка ожидаем:
\begin{equation*}
  \frac{n_{i+1}}{n_{i}}=\left(100\right)^{1/5}\simeq2.512, \quad \frac{error_{i}}{error_{i+1}}=\left(100\right)^{(1-1/5)}\simeq39.811.
\end{equation*}

Из табличных данных следует
\begin{equation*}
  \frac{\left.n\right\vert_{\varepsilon=10^{-9}}}{\left.n\right\vert_{\varepsilon=10^{-7}}}=2.348, \quad
  \frac{\left.n\right\vert_{\varepsilon=10^{-11}}}{\left.n\right\vert_{\varepsilon=10^{-9}}}=2.463.
\end{equation*}
\begin{equation*}
  \frac{\left.error\right\vert_{\varepsilon=10^{-7}}}{\left.error\right\vert_{\varepsilon=10^{-9}}}=37.892, \quad
  \frac{\left.error\right\vert_{\varepsilon=10^{-9}}}{\left.error\right\vert_{\varepsilon=10^{-11}}}=38.584,
\end{equation*}
что неплохо согласуется с теорией.

\subsection*{График решения}
\begin{tikzpicture}
  \begin{axis}[width=15cm, height=8cm, legend pos=outer north east, grid = major, grid style={dashed, gray!30}]
    \addplot [thick, mark = none, red]table [x=t, y=x1, col sep=comma] {data/points.csv};
    \addlegendentry{$x_1$}
  \end{axis}
\end{tikzpicture}
\begin{tikzpicture}
  \begin{axis}[width=15cm, height=8cm, legend pos=outer north east, grid = major, grid style={dashed, gray!30}]
    \addplot [thick, mark = none, blue]table [x=t, y=x2, col sep=comma] {data/points.csv};
    \addlegendentry{$x_2$}
  \end{axis}
\end{tikzpicture}

% \addcontentsline{toc}{section}{Список литературы}
% \begin{thebibliography}{99}
%   \bibitem{RKM}\textbf{О.Б. Арушанян, С.Ф. Залеткин} Решение систем обыкновенных дифференциальных уравнений методами Рунге--Кутта.
%   \bibitem{TAC}\textbf{Д.П. Ким} Теория автоматического управления. Том 2.
%   \bibitem{OM}\textbf{Н.Л. Майорова, Д.В. Глазков} Методы оптимизации.
%   \bibitem{DEP}\textbf{А.Ф. Филиппов} Сборник задач по дифференциальным уравнениям.
% \end{thebibliography}
\end{document}