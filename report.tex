\documentclass[a4paper,12pt]{article}
\usepackage[T2A]{fontenc}
\usepackage[utf8]{inputenc}
\usepackage[russian]{babel}
\usepackage[style=russian]{csquotes}
\usepackage{xcolor}
\usepackage{amsmath}
\usepackage{tikz}
\usepackage{pgfplots}
\usepackage[l3]{csvsimple}

\usepackage{cleveref}
\crefformat{equation}{(#2#1#3)}
\crefrangeformat{equation}{(#3#1#4)~-~(#5#2#6)}
\crefmultiformat{equation}{(#2#1#3)}{ и~(#2#1#3)}{, (#2#1#3)}{ и~(#2#1#3)}

\allowdisplaybreaks

\patchcmd{\thebibliography}{\section*{Список литературы}}{}{}{}

\newcommand{\UpdateMe}[1]{\textcolor{red}{#1}}
\DeclareMathOperator*{\argmax}{argmax}

\newcommand{\University}{Московский государственный университет имени М.~В.~Ломоносова}
\newcommand{\Department}{Кафедра \UpdateMe{НАЗВАНИЕ-КАФЕДРЫ}}
\newcommand{\Student}{\UpdateMe{ИМЯ-СТУДЕНТА}}
\newcommand{\GroupNum}{\UpdateMe{НОМЕР}}
\newcommand{\Seminar}{\UpdateMe{НАЗВАНИЕ-ПРАКТИКУМА}}

\begin{document}
% \begin{titlepage}
%   \centering
%   {\scshape\Large \University\par}\vspace{1cm}{\scshape\large \Department\par}
%   \vfill
%   {\huge\bfseries ОТЧЕТ\par}{\Largeпо задаче практикума \enquote{\Seminar}\par}
%   \vfill
%   \hfill\begin{minipage}{0.45\linewidth}Выполнил студент гр. \GroupNum:\\\Student\end{minipage}
%   \vfill
%   {\large Москва, \the\year{}\par}
% \end{titlepage}

\section*{Задание}
Найти решение задачи, удовлетворяющее указанным условиям ({\bfseries Задача~2.30}):
\begin{gather*}
  \frac{d^2x}{dt^2}=\frac{x}{1+t^4}, \quad t\in(0, +\infty),\\
  \left.x\right\vert_{t\to\infty}\to1.
\end{gather*}

\section*{Постановка задачи}

Введем обозначения \(x_1 = x, x_2=\dot{x}\). Тогда задача перепишется в виде:
\begin{gather}
  \label{common_system} \begin{cases}\dot{x}_2=\frac{x_1}{1+t^4}\\ \dot{x}_1=x_2 \end{cases},\\
  \label{exp_right_condition} x_1(\infty)=1.
\end{gather}
Последнее условие неявно влечет за собой требование
\begin{equation}
  \label{imp_right_condition} x_2(\infty)=0.
\end{equation}
\section*{Методика решения}

\subsection*{Метод Рунге-Кутта}

Пусть дана задача Коши
\begin{equation*}
  \begin{cases} \dot{x}_1=f_1(x_1,x_2,t)\\ \dot{x}_2=f_2(x_1,x_2,t) \end{cases}
\end{equation*}
с известными в начальный момент времени значениями \(x_1^0\) и \(\ x_2^0\).

Допустим уже найдены значения в точке \(t_i\): \(x_1^{i}\) и \(\ x_2^{i}\). Тогда значения в точке \(t_{i+1}\) вычисляются по
схеме:

\begin{gather*}
  k_{m1}=h f_m(x_1^{i},x_2^{i}, t), \\ 
  k_{m2}=h f_m(x_1^{i} + 0.5 k_{11},x_2^{i} + 0.5 k_{21}, t + 0.5 h), \\ 
  k_{m3}=h f_m(x_1^{i} + 0.25 (k_{11}+k_{12}),x_2^{i} + 0.25 (k_{21}+k_{22}), t + 0.5 h),\\
  k_{m4}=f_m(x_1^{i} - k_{12} + 2 k_{13},x_2^{i} - k_{22} + 2 k_{23}, t + h),\\
  \begin{align*}
    k_{m5}=h f_m(x_1^{i} + \frac{1}{27}(7 k_{11}+10 k_{12}+k_{14}), x_2^{i} + \frac{1}{27}(7 k_{21}+10 k_{22}+k_{24}), t + \frac{2}{3} h),
  \end{align*} \\
  \begin{align*}
    k_{m6}=h f_m( & x_1^{i} + \frac{1}{625}(28 k_{11}-125 k_{12}+ 546 k_{13}+54 k_{14}-378 k_{15}),                      \\
                  & x_2^{i} + \frac{1}{625}(28 k_{21}-125 k_{22}+ 546 k_{23}+54 k_{24}-378 k_{25}), t + \frac{1}{5} h),
  \end{align*} \\
  x_1^{i+1}=\gamma_1^h(x_1^{i})=x_1^{i} + \frac{1}{24}k_{11} + \frac{5}{48} k_{14} + \frac{27}{56} k_{15} + \frac{125}{336} k_{16}, \\ 
  x_2^{i+1}=\gamma_2^h(x_2^{i})=x_2^{i} + \frac{1}{24}k_{21} + \frac{5}{48} k_{24} + \frac{27}{56} k_{25} + \frac{125}{336} k_{26},
\end{gather*}

где \(h=h(t_i)\) - адаптивный шаг по времени.

\subsection*{Адаптивный шаг по времени}

Пусть задан уровень погрешности \(\varepsilon\). Обозначим за \(y^{i}\)
значение искомой функции в текущей точке, а за \(h\) - текущий шаг по
времени.

Погрешность \(m\)-ой функции \(\gamma_{m}\) вычисляется с помощью контрольного члена:
\begin{equation*}
  \delta\gamma_{m} = \frac{-1}{336}\left(42 k_{m1}+224k_{m3}+21k_{m4}-162 k_{m5}-125 k_{m6}\right)
\end{equation*}
Тогда новый шаг на каждой итерации будет вычисляться по формулам
\begin{equation*}
  err = \Vert (\delta\gamma_{1}, \delta\gamma_{2})\Vert,
\end{equation*}
\begin{equation*}
  h'=\begin{cases}
    0.95 h \sqrt[5]{\frac{\varepsilon}{x+err}}, \quad err \not\in [0.1 \varepsilon, \varepsilon] \\
    h, \quad  err \in [0.1 \varepsilon, \varepsilon]
  \end{cases},
\end{equation*}
где \(x\) - маленькая
положительная константа избавляющая нас от необходимости проверять, что
\(err \neq 0\). Например можно взять \(x=0.001 \varepsilon\).

\subsection*{Критерий остановки}

Первое уравнение системы \cref{common_system} при \(t\to\infty\) примет вид
\begin{equation}
  \label{stop_cond} \left.\dot{x}_2\right\vert_{t\to\infty}=\frac{1}{\infty}\to 0.
\end{equation}

При заданном уровне погрешности \(\varepsilon\) мы можем принять, что момент времени \(t_e=t^N\) "равносилен" бесконечности, при условии:
\begin{equation*}
  \frac{x^N_1}{1+\left(t^N\right)^4} < \varepsilon
\end{equation*}


\subsection*{Метод стрельбы}

Для краевой задачи \cref{common_system} выберем начальные значения неизвестных функций в качестве параметров пристрелки

\begin{equation*}
  x_1^0=a, \quad x_2^0=b
\end{equation*}

Тогда решая систему \cref{common_system} с такими начальными условиями до момента времени \(t=t_e\) получим некоторое решение \(\tilde{x}_1\) и \(\tilde{x}_2\).

Варьируя параметры \(a\) и \(b\) необходимо добиться выполнения краевых условий \cref{exp_right_condition}, \cref{imp_right_condition}:
\begin{equation*}
  \tilde{x}_1(t_e)=x_1^N=1, \quad \tilde{x}_2(t_e)=x_2^N=0
\end{equation*}

Будем искать параметры \(a\) и \(b\) методом простых итераций. Введём
функции \(\phi_1\) и \(\phi_2\) показывающие отклонение от заданного
граничного условия:
\begin{gather*}
  \phi_1(a,b)=x_1^N-1=0\\
  \phi_2(a,b)=_2^N-0=0
\end{gather*}

Разложив эти функции в окрестности точки \((a^i, b^i)\) получим:

\begin{gather*}
  \phi_1(a,b)\simeq\phi_1(a^i,b^i)+\delta a\frac{\partial \phi_1}{\partial a}(a^i,b^i)+\delta b\frac{\partial \phi_1}{\partial b}(a^i,b^i)\simeq0\\
  \phi_2(a,b)\simeq\phi_2(a^i,b^i)+\delta a\frac{\partial \phi_2}{\partial a}(a^i,b^i)+\delta b\frac{\partial \phi_2}{\partial b}(a^i,b^i)\simeq0
\end{gather*}

где \(\delta z=z^{i+1}-z^{i}\). В матричном виде уравнения примут вид

\begin{equation*}
  \begin{pmatrix} 
    \frac{\partial \phi_1}{\partial a}(a^i,b^i)&\frac{\partial \phi_1}{\partial b}(a^i,b^i)\\
    \frac{\partial \phi_2}{\partial a}(a^i,b^i)&\frac{\partial \phi_2}{\partial b}(a^i,b^i)
  \end{pmatrix} \cdot \begin{pmatrix}
    \delta a\\
    \delta b
  \end{pmatrix} =W\cdot\begin{pmatrix}
    \delta a\\
    \delta b
  \end{pmatrix} = -\begin{pmatrix}
    \phi_1(a^i,b^i)\\
    \phi_2(a^i,b^i)
  \end{pmatrix}
\end{equation*}

Тогда итерационный процесс может быть выражен формулой

\begin{equation*}
  \begin{pmatrix}
    a^{i+1}\\
    b^{i+1}
  \end{pmatrix}=\begin{pmatrix}
    a^{i}\\
    b^{i}
  \end{pmatrix}-W^{-1}\cdot \begin{pmatrix}
    \phi_1(a^i,b^i)\\
    \phi_2(a^i,b^i)
  \end{pmatrix}
\end{equation*}

В случае \(\det\vert W\vert=0\) значения сбрасывались на случайные из интервала \([-10, 10]\).

\section*{Результаты}

\csvautotabular{stats.csv}

С учетом используемого шага погрешности должно быть выполнено следующее:
\begin{gather*}
  \frac{n_{i+1}}{n_{i}}=\left(\frac{\varepsilon_{i+1}}{\varepsilon_{i}}\right)^{1/s}=\left(100\right)^{1/5}\simeq2.512, \\
  \frac{error_{i}}{error_{i+1}}=\left(\frac{\varepsilon_{i}}{\varepsilon_{i+1}}\right)^{(1-1/s)}=\left(100\right)^{(1-1/5)}\simeq39.811,
\end{gather*}
где \(s\) - порядок метода Рунге--Кутта.

\begin{tikzpicture}
  \begin{axis}[width=15cm, height=8cm, legend pos=outer north east, grid = major, grid style={dashed, gray!30}]
    \addplot [thick, mark = none, red]table [x=t, y=x1, col sep=comma] {data.csv};
    \addlegendentry{$x_1$}
    \addplot [thick, mark = none, blue]table [x=t, y=x2, col sep=comma] {data.csv};
    \addlegendentry{$x_2$}
  \end{axis}
\end{tikzpicture}
\begin{equation*}
  \frac{\left.n\right\vert_{\varepsilon=10^{-7}}}{\left.n\right\vert_{\varepsilon=10^{-5}}}=2.708, \quad
  \frac{\left.n\right\vert_{\varepsilon=10^{-9}}}{\left.n\right\vert_{\varepsilon=10^{-7}}}=2.569.
\end{equation*}
\begin{equation*}
  \frac{\left.error\right\vert_{\varepsilon=10^{-5}}}{\left.error\right\vert_{\varepsilon=10^{-7}}}=44.422, \quad
  \frac{\left.error\right\vert_{\varepsilon=10^{-7}}}{\left.error\right\vert_{\varepsilon=10^{-9}}}=13.125.
\end{equation*}

% \addcontentsline{toc}{section}{Список литературы}
% \begin{thebibliography}{99}
%   \bibitem{RKM}\textbf{О.Б. Арушанян, С.Ф. Залеткин} Решение систем обыкновенных дифференциальных уравнений методами Рунге--Кутта.
%   \bibitem{TAC}\textbf{Д.П. Ким} Теория автоматического управления. Том 2.
%   \bibitem{OM}\textbf{Н.Л. Майорова, Д.В. Глазков} Методы оптимизации.
%   \bibitem{DEP}\textbf{А.Ф. Филиппов} Сборник задач по дифференциальным уравнениям.
% \end{thebibliography}
\end{document}