\documentclass[a4paper,12pt]{article}
\usepackage[T2A]{fontenc}
\usepackage[utf8]{inputenc}
\usepackage[russian]{babel}
\usepackage[style=russian]{csquotes}
\usepackage{xcolor}
\usepackage{amsmath}
\usepackage{tikz}
\usepackage{pgfplots}
\usepackage[l3]{csvsimple}

\usepackage{cleveref}
\crefformat{equation}{(#2#1#3)}
\crefrangeformat{equation}{(#3#1#4)~-~(#5#2#6)}
\crefmultiformat{equation}{(#2#1#3)}{ и~(#2#1#3)}{, (#2#1#3)}{ и~(#2#1#3)}

\allowdisplaybreaks

\patchcmd{\thebibliography}{\section*{Список литературы}}{}{}{}

\newcommand{\UpdateMe}[1]{\textcolor{red}{#1}}
\DeclareMathOperator*{\argmax}{argmax}

\newcommand{\University}{Московский государственный университет имени М.~В.~Ломоносова}
\newcommand{\Department}{Кафедра \UpdateMe{НАЗВАНИЕ-КАФЕДРЫ}}
\newcommand{\Student}{\UpdateMe{ИМЯ-СТУДЕНТА}}
\newcommand{\GroupNum}{\UpdateMe{НОМЕР}}
\newcommand{\Seminar}{\UpdateMe{НАЗВАНИЕ-ПРАКТИКУМА}}

\begin{document}
\begin{titlepage}
    \centering
    {\scshape\Large \University\par}\vspace{1cm}{\scshape\large \Department\par}
    \vfill
    {\huge\bfseries ОТЧЕТ\par}{\Largeпо задаче практикума \enquote{\Seminar}\par}
    \vfill
    \hfill\begin{minipage}{0.45\linewidth}Выполнил студент гр. \GroupNum:\\\Student\end{minipage}
    \vfill
    {\large Москва, \the\year{}\par}
\end{titlepage}

\section*{Задание}
\UpdateMe{Описание задачи}

\section*{Теория}
\UpdateMe{Обозначения, основные конструкции, мат. постановка задачи}

\section*{Результаты}
\UpdateMe{Полученные результаты}

\addcontentsline{toc}{section}{Список литературы}
\begin{thebibliography}{99}
  \bibitem{RKM}\textbf{О.Б. Арушанян, С.Ф. Залеткин} Решение систем обыкновенных дифференциальных уравнений методами Рунге--Кутта.
  \bibitem{TAC}\textbf{Д.П. Ким} Теория автоматического управления. Том 2.
  \bibitem{OM}\textbf{Н.Л. Майорова, Д.В. Глазков} Методы оптимизации.
  \bibitem{DEP}\textbf{А.Ф. Филиппов} Сборник задач по дифференциальным уравнениям.
\end{thebibliography}
\end{document}