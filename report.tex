\documentclass[a4paper,12pt]{article}
\usepackage[T2A]{fontenc}
\usepackage[utf8]{inputenc}
\usepackage[russian]{babel}
\usepackage[style=russian]{csquotes}
\usepackage{xcolor}
\usepackage{amsmath}
\usepackage{tikz}
\usepackage{pgfplots}
\usepackage[l3]{csvsimple}

\usepackage{cleveref}
\crefformat{equation}{(#2#1#3)}
\crefrangeformat{equation}{(#3#1#4)~-~(#5#2#6)}
\crefmultiformat{equation}{(#2#1#3)}{ и~(#2#1#3)}{, (#2#1#3)}{ и~(#2#1#3)}

\allowdisplaybreaks

\patchcmd{\thebibliography}{\section*{Список литературы}}{}{}{}

\newcommand{\UpdateMe}[1]{\textcolor{red}{#1}}
\DeclareMathOperator*{\argmax}{argmax}

\newcommand{\University}{Московский государственный университет имени М.~В.~Ломоносова}
\newcommand{\Department}{Кафедра \UpdateMe{НАЗВАНИЕ-КАФЕДРЫ}}
\newcommand{\Student}{\UpdateMe{ИМЯ-СТУДЕНТА}}
\newcommand{\GroupNum}{\UpdateMe{НОМЕР}}
\newcommand{\Seminar}{\UpdateMe{НАЗВАНИЕ-ПРАКТИКУМА}}

\begin{document}
% \begin{titlepage}
%     \centering
%     {\scshape\Large \University\par}\vspace{1cm}{\scshape\large \Department\par}
%     \vfill
%     {\huge\bfseries ОТЧЕТ\par}{\Largeпо задаче практикума \enquote{\Seminar}\par}
%     \vfill
%     \hfill\begin{minipage}{0.45\linewidth}Выполнил студент гр. \GroupNum:\\\Student\end{minipage}
%     \vfill
%     {\large Москва, \the\year{}\par}
% \end{titlepage}

\section*{Задание}
Найти корни уравнения (Задача 1.21):
\begin{equation} \label{eq}
  f(x)=\sum_{0}^{3}\frac{(\pi-k)^2}{k!}x^{2k+1}=0.1\alpha
\end{equation}

\section*{Анализ уравнения}
Левая часть уравнения \cref{eq} является монотонной функций, т.к. её производная не отрицательна
\begin{equation}
  f'(x)=\sum_{0}^{3}\frac{(\pi-k)^2}{k!}(2k+1)x^{2k} \ge 0
\end{equation}
Следовательно уравнение имеет единственный корень.


\section*{Метод простой итерации}
Для поиска корня применим метод простой итерации, но используем два разных способа в зависимости от \(\alpha\).
\subsection*{Случай \(\vert\alpha\vert\le100\)}
В этом случае приведем уравнение \cref{eq} к виду
\begin{gather}
  x = \phi_1(x), \quad \phi_1(x) = \frac{1}{\pi^2}\left(0.1 \alpha - \sum_{1}^{3}\frac{(\pi-k)^2}{k!}x^{2k+1}\right),\\
  \phi_1'(x) = \frac{1}{\pi^2}\left(- \sum_{1}^{3}\frac{(\pi-k)^2}{k!}(2k+1)x^{2k}\right).
\end{gather}
При указанных значениях \(\alpha\) в окрестности корня выполнено условие \(\phi_1'(x) \le 1\) и метод сходится.

\subsection*{Случай \(\vert\alpha\vert > 100\)}
Введем функцию ошибки \(g(x)\):
\begin{equation}
  g(x)=0.1\alpha - f(x)
\end{equation}
Итерационный процесс построим по следующей схеме:
\begin{gather}
  x = \phi_2(x), \quad \phi_2(x)=x + k(\alpha)g(x),\\
  \phi_2'(x) = 1 - k(\alpha) f'(x).
\end{gather}
Условие сходимости достигается за счёт выбора функции \(k(\alpha)\). Например её можно взять кусочно-постоянной:
\begin{equation}
  k(\alpha) = 0.5\times 10^{\lfloor \log_{10}(\alpha)\rfloor}
\end{equation}

\section*{Результаты}
\csvautotabular{data/stats.csv}

% \addcontentsline{toc}{section}{Список литературы}
% \begin{thebibliography}{99}
%   \bibitem{RKM}\textbf{О.Б. Арушанян, С.Ф. Залеткин} Решение систем обыкновенных дифференциальных уравнений методами Рунге--Кутта.
%   \bibitem{TAC}\textbf{Д.П. Ким} Теория автоматического управления. Том 2.
%   \bibitem{OM}\textbf{Н.Л. Майорова, Д.В. Глазков} Методы оптимизации.
%   \bibitem{DEP}\textbf{А.Ф. Филиппов} Сборник задач по дифференциальным уравнениям.
% \end{thebibliography}
\end{document}